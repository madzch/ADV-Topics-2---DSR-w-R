% Options for packages loaded elsewhere
\PassOptionsToPackage{unicode}{hyperref}
\PassOptionsToPackage{hyphens}{url}
%
\documentclass[
]{article}
\usepackage{amsmath,amssymb}
\usepackage{iftex}
\ifPDFTeX
  \usepackage[T1]{fontenc}
  \usepackage[utf8]{inputenc}
  \usepackage{textcomp} % provide euro and other symbols
\else % if luatex or xetex
  \usepackage{unicode-math} % this also loads fontspec
  \defaultfontfeatures{Scale=MatchLowercase}
  \defaultfontfeatures[\rmfamily]{Ligatures=TeX,Scale=1}
\fi
\usepackage{lmodern}
\ifPDFTeX\else
  % xetex/luatex font selection
\fi
% Use upquote if available, for straight quotes in verbatim environments
\IfFileExists{upquote.sty}{\usepackage{upquote}}{}
\IfFileExists{microtype.sty}{% use microtype if available
  \usepackage[]{microtype}
  \UseMicrotypeSet[protrusion]{basicmath} % disable protrusion for tt fonts
}{}
\makeatletter
\@ifundefined{KOMAClassName}{% if non-KOMA class
  \IfFileExists{parskip.sty}{%
    \usepackage{parskip}
  }{% else
    \setlength{\parindent}{0pt}
    \setlength{\parskip}{6pt plus 2pt minus 1pt}}
}{% if KOMA class
  \KOMAoptions{parskip=half}}
\makeatother
\usepackage{xcolor}
\usepackage[margin=1in]{geometry}
\usepackage{color}
\usepackage{fancyvrb}
\newcommand{\VerbBar}{|}
\newcommand{\VERB}{\Verb[commandchars=\\\{\}]}
\DefineVerbatimEnvironment{Highlighting}{Verbatim}{commandchars=\\\{\}}
% Add ',fontsize=\small' for more characters per line
\usepackage{framed}
\definecolor{shadecolor}{RGB}{248,248,248}
\newenvironment{Shaded}{\begin{snugshade}}{\end{snugshade}}
\newcommand{\AlertTok}[1]{\textcolor[rgb]{0.94,0.16,0.16}{#1}}
\newcommand{\AnnotationTok}[1]{\textcolor[rgb]{0.56,0.35,0.01}{\textbf{\textit{#1}}}}
\newcommand{\AttributeTok}[1]{\textcolor[rgb]{0.13,0.29,0.53}{#1}}
\newcommand{\BaseNTok}[1]{\textcolor[rgb]{0.00,0.00,0.81}{#1}}
\newcommand{\BuiltInTok}[1]{#1}
\newcommand{\CharTok}[1]{\textcolor[rgb]{0.31,0.60,0.02}{#1}}
\newcommand{\CommentTok}[1]{\textcolor[rgb]{0.56,0.35,0.01}{\textit{#1}}}
\newcommand{\CommentVarTok}[1]{\textcolor[rgb]{0.56,0.35,0.01}{\textbf{\textit{#1}}}}
\newcommand{\ConstantTok}[1]{\textcolor[rgb]{0.56,0.35,0.01}{#1}}
\newcommand{\ControlFlowTok}[1]{\textcolor[rgb]{0.13,0.29,0.53}{\textbf{#1}}}
\newcommand{\DataTypeTok}[1]{\textcolor[rgb]{0.13,0.29,0.53}{#1}}
\newcommand{\DecValTok}[1]{\textcolor[rgb]{0.00,0.00,0.81}{#1}}
\newcommand{\DocumentationTok}[1]{\textcolor[rgb]{0.56,0.35,0.01}{\textbf{\textit{#1}}}}
\newcommand{\ErrorTok}[1]{\textcolor[rgb]{0.64,0.00,0.00}{\textbf{#1}}}
\newcommand{\ExtensionTok}[1]{#1}
\newcommand{\FloatTok}[1]{\textcolor[rgb]{0.00,0.00,0.81}{#1}}
\newcommand{\FunctionTok}[1]{\textcolor[rgb]{0.13,0.29,0.53}{\textbf{#1}}}
\newcommand{\ImportTok}[1]{#1}
\newcommand{\InformationTok}[1]{\textcolor[rgb]{0.56,0.35,0.01}{\textbf{\textit{#1}}}}
\newcommand{\KeywordTok}[1]{\textcolor[rgb]{0.13,0.29,0.53}{\textbf{#1}}}
\newcommand{\NormalTok}[1]{#1}
\newcommand{\OperatorTok}[1]{\textcolor[rgb]{0.81,0.36,0.00}{\textbf{#1}}}
\newcommand{\OtherTok}[1]{\textcolor[rgb]{0.56,0.35,0.01}{#1}}
\newcommand{\PreprocessorTok}[1]{\textcolor[rgb]{0.56,0.35,0.01}{\textit{#1}}}
\newcommand{\RegionMarkerTok}[1]{#1}
\newcommand{\SpecialCharTok}[1]{\textcolor[rgb]{0.81,0.36,0.00}{\textbf{#1}}}
\newcommand{\SpecialStringTok}[1]{\textcolor[rgb]{0.31,0.60,0.02}{#1}}
\newcommand{\StringTok}[1]{\textcolor[rgb]{0.31,0.60,0.02}{#1}}
\newcommand{\VariableTok}[1]{\textcolor[rgb]{0.00,0.00,0.00}{#1}}
\newcommand{\VerbatimStringTok}[1]{\textcolor[rgb]{0.31,0.60,0.02}{#1}}
\newcommand{\WarningTok}[1]{\textcolor[rgb]{0.56,0.35,0.01}{\textbf{\textit{#1}}}}
\usepackage{graphicx}
\makeatletter
\def\maxwidth{\ifdim\Gin@nat@width>\linewidth\linewidth\else\Gin@nat@width\fi}
\def\maxheight{\ifdim\Gin@nat@height>\textheight\textheight\else\Gin@nat@height\fi}
\makeatother
% Scale images if necessary, so that they will not overflow the page
% margins by default, and it is still possible to overwrite the defaults
% using explicit options in \includegraphics[width, height, ...]{}
\setkeys{Gin}{width=\maxwidth,height=\maxheight,keepaspectratio}
% Set default figure placement to htbp
\makeatletter
\def\fps@figure{htbp}
\makeatother
\setlength{\emergencystretch}{3em} % prevent overfull lines
\providecommand{\tightlist}{%
  \setlength{\itemsep}{0pt}\setlength{\parskip}{0pt}}
\setcounter{secnumdepth}{-\maxdimen} % remove section numbering
\ifLuaTeX
  \usepackage{selnolig}  % disable illegal ligatures
\fi
\usepackage{bookmark}
\IfFileExists{xurl.sty}{\usepackage{xurl}}{} % add URL line breaks if available
\urlstyle{same}
\hypersetup{
  pdftitle={D2SC ICMA notebook},
  pdfauthor={Madison Chin},
  hidelinks,
  pdfcreator={LaTeX via pandoc}}

\title{D2SC ICMA notebook}
\author{Madison Chin}
\date{2024-10-30}

\begin{document}
\maketitle

{
\setcounter{tocdepth}{2}
\tableofcontents
}
\#ICMA Sept 9 2024

\begin{Shaded}
\begin{Highlighting}[]
\CommentTok{\#The downloaded binary packages are in}
\CommentTok{\#C:\textbackslash{}Users\textbackslash{}mchin\textbackslash{}AppData\textbackslash{}Local\textbackslash{}Temp\textbackslash{}RtmpW0fCpq\textbackslash{}downloaded\_packages}
\FunctionTok{plot}\NormalTok{(cars)}
\end{Highlighting}
\end{Shaded}

\includegraphics{r-notebook1-412_files/figure-latex/unnamed-chunk-1-1.pdf}
\#ICMA Sept 11 2024

\begin{Shaded}
\begin{Highlighting}[]
\NormalTok{ohno\_this\_is\_a\_nightmare }\OtherTok{\textless{}{-}} \StringTok{"hello"}
\end{Highlighting}
\end{Shaded}

\begin{Shaded}
\begin{Highlighting}[]
\DecValTok{2}\SpecialCharTok{+}\DecValTok{2}
\end{Highlighting}
\end{Shaded}

\begin{verbatim}
## [1] 4
\end{verbatim}

\begin{Shaded}
\begin{Highlighting}[]
\NormalTok{x }\OtherTok{\textless{}{-}} \DecValTok{2}\SpecialCharTok{+}\DecValTok{2}
\end{Highlighting}
\end{Shaded}

\begin{Shaded}
\begin{Highlighting}[]
\NormalTok{?mean}
\end{Highlighting}
\end{Shaded}

\begin{verbatim}
## starting httpd help server ... done
\end{verbatim}

\begin{Shaded}
\begin{Highlighting}[]
\CommentTok{\#how we can look for things we don\textquotesingle{}t know.(use ?)}
\CommentTok{\#function call looks like mean()}
\CommentTok{\#arrays look like c(2,3,4)}
\end{Highlighting}
\end{Shaded}

\begin{Shaded}
\begin{Highlighting}[]
\FunctionTok{sd}\NormalTok{(}\DecValTok{1}\SpecialCharTok{:}\DecValTok{2}\NormalTok{) }\SpecialCharTok{\^{}} \DecValTok{2}
\end{Highlighting}
\end{Shaded}

\begin{verbatim}
## [1] 0.5
\end{verbatim}

\#ICMA Sept 16 2024

\begin{Shaded}
\begin{Highlighting}[]
\CommentTok{\#make sure to complete first code chunks }
\CommentTok{\# to create a vector, use c(list, numbers, inside, like, this)}
\NormalTok{my\_numbers }\OtherTok{\textless{}{-}} \FunctionTok{c}\NormalTok{( }\DecValTok{1}\NormalTok{,}\DecValTok{2}\NormalTok{,}\DecValTok{3}\NormalTok{,}\DecValTok{4}\NormalTok{,}\DecValTok{5}\NormalTok{,}\DecValTok{8}\NormalTok{,}\DecValTok{9}\NormalTok{,}\DecValTok{10}\NormalTok{)}
\end{Highlighting}
\end{Shaded}

\begin{Shaded}
\begin{Highlighting}[]
\FunctionTok{library}\NormalTok{(magrittr)}
\FunctionTok{library}\NormalTok{(tidyverse)}
\end{Highlighting}
\end{Shaded}

\begin{verbatim}
## -- Attaching core tidyverse packages ------------------------ tidyverse 2.0.0 --
## v dplyr     1.1.4     v readr     2.1.5
## v forcats   1.0.0     v stringr   1.5.1
## v ggplot2   3.5.1     v tibble    3.2.1
## v lubridate 1.9.3     v tidyr     1.3.1
## v purrr     1.0.2     
## -- Conflicts ------------------------------------------ tidyverse_conflicts() --
## x tidyr::extract()   masks magrittr::extract()
## x dplyr::filter()    masks stats::filter()
## x dplyr::lag()       masks stats::lag()
## x purrr::set_names() masks magrittr::set_names()
## i Use the conflicted package (<http://conflicted.r-lib.org/>) to force all conflicts to become errors
\end{verbatim}

\begin{Shaded}
\begin{Highlighting}[]
\CommentTok{\#must be loaded each session}
\end{Highlighting}
\end{Shaded}

\begin{Shaded}
\begin{Highlighting}[]
\FunctionTok{mean}\NormalTok{(my\_numbers)}
\end{Highlighting}
\end{Shaded}

\begin{verbatim}
## [1] 5.25
\end{verbatim}

\begin{Shaded}
\begin{Highlighting}[]
\NormalTok{my\_numbers }\SpecialCharTok{\%\textgreater{}\%} 
  \FunctionTok{mean}\NormalTok{()}
\end{Highlighting}
\end{Shaded}

\begin{verbatim}
## [1] 5.25
\end{verbatim}

\begin{Shaded}
\begin{Highlighting}[]
\FunctionTok{getwd}\NormalTok{()}
\end{Highlighting}
\end{Shaded}

\begin{verbatim}
## [1] "C:/Users/mchin/OneDrive/Documents/GitHub/ADV-Topics-2---DSR-w-R/ICMAs"
\end{verbatim}

\begin{Shaded}
\begin{Highlighting}[]
\NormalTok{?mtcars}
\NormalTok{mtcars}
\end{Highlighting}
\end{Shaded}

\begin{verbatim}
##                      mpg cyl  disp  hp drat    wt  qsec vs am gear carb
## Mazda RX4           21.0   6 160.0 110 3.90 2.620 16.46  0  1    4    4
## Mazda RX4 Wag       21.0   6 160.0 110 3.90 2.875 17.02  0  1    4    4
## Datsun 710          22.8   4 108.0  93 3.85 2.320 18.61  1  1    4    1
## Hornet 4 Drive      21.4   6 258.0 110 3.08 3.215 19.44  1  0    3    1
## Hornet Sportabout   18.7   8 360.0 175 3.15 3.440 17.02  0  0    3    2
## Valiant             18.1   6 225.0 105 2.76 3.460 20.22  1  0    3    1
## Duster 360          14.3   8 360.0 245 3.21 3.570 15.84  0  0    3    4
## Merc 240D           24.4   4 146.7  62 3.69 3.190 20.00  1  0    4    2
## Merc 230            22.8   4 140.8  95 3.92 3.150 22.90  1  0    4    2
## Merc 280            19.2   6 167.6 123 3.92 3.440 18.30  1  0    4    4
## Merc 280C           17.8   6 167.6 123 3.92 3.440 18.90  1  0    4    4
## Merc 450SE          16.4   8 275.8 180 3.07 4.070 17.40  0  0    3    3
## Merc 450SL          17.3   8 275.8 180 3.07 3.730 17.60  0  0    3    3
## Merc 450SLC         15.2   8 275.8 180 3.07 3.780 18.00  0  0    3    3
## Cadillac Fleetwood  10.4   8 472.0 205 2.93 5.250 17.98  0  0    3    4
## Lincoln Continental 10.4   8 460.0 215 3.00 5.424 17.82  0  0    3    4
## Chrysler Imperial   14.7   8 440.0 230 3.23 5.345 17.42  0  0    3    4
## Fiat 128            32.4   4  78.7  66 4.08 2.200 19.47  1  1    4    1
## Honda Civic         30.4   4  75.7  52 4.93 1.615 18.52  1  1    4    2
## Toyota Corolla      33.9   4  71.1  65 4.22 1.835 19.90  1  1    4    1
## Toyota Corona       21.5   4 120.1  97 3.70 2.465 20.01  1  0    3    1
## Dodge Challenger    15.5   8 318.0 150 2.76 3.520 16.87  0  0    3    2
## AMC Javelin         15.2   8 304.0 150 3.15 3.435 17.30  0  0    3    2
## Camaro Z28          13.3   8 350.0 245 3.73 3.840 15.41  0  0    3    4
## Pontiac Firebird    19.2   8 400.0 175 3.08 3.845 17.05  0  0    3    2
## Fiat X1-9           27.3   4  79.0  66 4.08 1.935 18.90  1  1    4    1
## Porsche 914-2       26.0   4 120.3  91 4.43 2.140 16.70  0  1    5    2
## Lotus Europa        30.4   4  95.1 113 3.77 1.513 16.90  1  1    5    2
## Ford Pantera L      15.8   8 351.0 264 4.22 3.170 14.50  0  1    5    4
## Ferrari Dino        19.7   6 145.0 175 3.62 2.770 15.50  0  1    5    6
## Maserati Bora       15.0   8 301.0 335 3.54 3.570 14.60  0  1    5    8
## Volvo 142E          21.4   4 121.0 109 4.11 2.780 18.60  1  1    4    2
\end{verbatim}

\begin{Shaded}
\begin{Highlighting}[]
\FunctionTok{write\_csv}\NormalTok{(mtcars, }\CommentTok{\# object name, }
         \StringTok{"mtcars\_fromR.csv"} \CommentTok{\#file name to save it"}
\NormalTok{         )}
\CommentTok{\#utils is BaseR, but we want to use the \_ version, as it is a part of the tidyverse pkg.}
\CommentTok{\#is wihout rowname implementation and use of "rowtoname function"}
\end{Highlighting}
\end{Shaded}

\begin{Shaded}
\begin{Highlighting}[]
\FunctionTok{read\_csv}\NormalTok{(}\StringTok{"mtcars\_fromR.csv"}\NormalTok{)}
\end{Highlighting}
\end{Shaded}

\begin{verbatim}
## Rows: 32 Columns: 11
## -- Column specification --------------------------------------------------------
## Delimiter: ","
## dbl (11): mpg, cyl, disp, hp, drat, wt, qsec, vs, am, gear, carb
## 
## i Use `spec()` to retrieve the full column specification for this data.
## i Specify the column types or set `show_col_types = FALSE` to quiet this message.
\end{verbatim}

\begin{verbatim}
## # A tibble: 32 x 11
##      mpg   cyl  disp    hp  drat    wt  qsec    vs    am  gear  carb
##    <dbl> <dbl> <dbl> <dbl> <dbl> <dbl> <dbl> <dbl> <dbl> <dbl> <dbl>
##  1  21       6  160    110  3.9   2.62  16.5     0     1     4     4
##  2  21       6  160    110  3.9   2.88  17.0     0     1     4     4
##  3  22.8     4  108     93  3.85  2.32  18.6     1     1     4     1
##  4  21.4     6  258    110  3.08  3.22  19.4     1     0     3     1
##  5  18.7     8  360    175  3.15  3.44  17.0     0     0     3     2
##  6  18.1     6  225    105  2.76  3.46  20.2     1     0     3     1
##  7  14.3     8  360    245  3.21  3.57  15.8     0     0     3     4
##  8  24.4     4  147.    62  3.69  3.19  20       1     0     4     2
##  9  22.8     4  141.    95  3.92  3.15  22.9     1     0     4     2
## 10  19.2     6  168.   123  3.92  3.44  18.3     1     0     4     4
## # i 22 more rows
\end{verbatim}

\#ICMA Sept 18 2024

\begin{Shaded}
\begin{Highlighting}[]
\NormalTok{?ChickWeight}
\FunctionTok{glimpse}\NormalTok{(ChickWeight)}
\end{Highlighting}
\end{Shaded}

\begin{verbatim}
## Rows: 578
## Columns: 4
## $ weight <dbl> 42, 51, 59, 64, 76, 93, 106, 125, 149, 171, 199, 205, 40, 49, 5~
## $ Time   <dbl> 0, 2, 4, 6, 8, 10, 12, 14, 16, 18, 20, 21, 0, 2, 4, 6, 8, 10, 1~
## $ Chick  <ord> 1, 1, 1, 1, 1, 1, 1, 1, 1, 1, 1, 1, 2, 2, 2, 2, 2, 2, 2, 2, 2, ~
## $ Diet   <fct> 1, 1, 1, 1, 1, 1, 1, 1, 1, 1, 1, 1, 1, 1, 1, 1, 1, 1, 1, 1, 1, ~
\end{verbatim}

\begin{Shaded}
\begin{Highlighting}[]
\NormalTok{ChickWeight }\SpecialCharTok{\%\textgreater{}\%}
  \FunctionTok{select}\NormalTok{(Chick,weight) }\SpecialCharTok{\%\textgreater{}\%}
  \FunctionTok{head}\NormalTok{ (}\AttributeTok{n=} \DecValTok{3}\NormalTok{)}
\end{Highlighting}
\end{Shaded}

\begin{verbatim}
##   Chick weight
## 1     1     42
## 2     1     51
## 3     1     59
\end{verbatim}

\begin{Shaded}
\begin{Highlighting}[]
\CommentTok{\#adding a new column called weight lbs}
\NormalTok{ChickWeight }\SpecialCharTok{\%\textgreater{}\%}
  \FunctionTok{mutate}\NormalTok{(}\AttributeTok{weight\_lbs =}\NormalTok{ weight}\SpecialCharTok{/}\FloatTok{453.6}\NormalTok{) }\SpecialCharTok{\%\textgreater{}\%}
\CommentTok{\#mutate always adds the columns to the end of your graphs}
\CommentTok{\#create summary table that has, \#of chicks, avg weight,avg lbs on Day 20.}
  \CommentTok{\#for getting each diet group, we can use (the below) }
 \FunctionTok{group\_by}\NormalTok{(Diet,Time)}\SpecialCharTok{\%\textgreater{}\%}
  \FunctionTok{summarise}\NormalTok{(}\AttributeTok{N=} \FunctionTok{n}\NormalTok{(), }\AttributeTok{mean\_wgt\_gm =} \FunctionTok{mean}\NormalTok{(weight),}
            \AttributeTok{mean\_wgt\_lbs =} \FunctionTok{mean}\NormalTok{(weight\_lbs)) }\SpecialCharTok{\%\textgreater{}\%}
            \FunctionTok{filter}\NormalTok{(Time }\SpecialCharTok{==} \DecValTok{20}\NormalTok{)}
\end{Highlighting}
\end{Shaded}

\begin{verbatim}
## `summarise()` has grouped output by 'Diet'. You can override using the
## `.groups` argument.
\end{verbatim}

\begin{verbatim}
## # A tibble: 4 x 5
## # Groups:   Diet [4]
##   Diet   Time     N mean_wgt_gm mean_wgt_lbs
##   <fct> <dbl> <int>       <dbl>        <dbl>
## 1 1        20    17        170.        0.376
## 2 2        20    10        206.        0.453
## 3 3        20    10        259.        0.571
## 4 4        20     9        234.        0.516
\end{verbatim}

\begin{Shaded}
\begin{Highlighting}[]
\FunctionTok{glimpse}\NormalTok{(mtcars)}
\end{Highlighting}
\end{Shaded}

\begin{verbatim}
## Rows: 32
## Columns: 11
## $ mpg  <dbl> 21.0, 21.0, 22.8, 21.4, 18.7, 18.1, 14.3, 24.4, 22.8, 19.2, 17.8,~
## $ cyl  <dbl> 6, 6, 4, 6, 8, 6, 8, 4, 4, 6, 6, 8, 8, 8, 8, 8, 8, 4, 4, 4, 4, 8,~
## $ disp <dbl> 160.0, 160.0, 108.0, 258.0, 360.0, 225.0, 360.0, 146.7, 140.8, 16~
## $ hp   <dbl> 110, 110, 93, 110, 175, 105, 245, 62, 95, 123, 123, 180, 180, 180~
## $ drat <dbl> 3.90, 3.90, 3.85, 3.08, 3.15, 2.76, 3.21, 3.69, 3.92, 3.92, 3.92,~
## $ wt   <dbl> 2.620, 2.875, 2.320, 3.215, 3.440, 3.460, 3.570, 3.190, 3.150, 3.~
## $ qsec <dbl> 16.46, 17.02, 18.61, 19.44, 17.02, 20.22, 15.84, 20.00, 22.90, 18~
## $ vs   <dbl> 0, 0, 1, 1, 0, 1, 0, 1, 1, 1, 1, 0, 0, 0, 0, 0, 0, 1, 1, 1, 1, 0,~
## $ am   <dbl> 1, 1, 1, 0, 0, 0, 0, 0, 0, 0, 0, 0, 0, 0, 0, 0, 0, 1, 1, 1, 0, 0,~
## $ gear <dbl> 4, 4, 4, 3, 3, 3, 3, 4, 4, 4, 4, 3, 3, 3, 3, 3, 3, 4, 4, 4, 3, 3,~
## $ carb <dbl> 4, 4, 1, 1, 2, 1, 4, 2, 2, 4, 4, 3, 3, 3, 4, 4, 4, 1, 2, 1, 1, 2,~
\end{verbatim}

\#ICMA Sept 23 2024

\begin{Shaded}
\begin{Highlighting}[]
\FunctionTok{head}\NormalTok{(billboard)}
\end{Highlighting}
\end{Shaded}

\begin{verbatim}
## # A tibble: 6 x 79
##   artist      track date.entered   wk1   wk2   wk3   wk4   wk5   wk6   wk7   wk8
##   <chr>       <chr> <date>       <dbl> <dbl> <dbl> <dbl> <dbl> <dbl> <dbl> <dbl>
## 1 2 Pac       Baby~ 2000-02-26      87    82    72    77    87    94    99    NA
## 2 2Ge+her     The ~ 2000-09-02      91    87    92    NA    NA    NA    NA    NA
## 3 3 Doors Do~ Kryp~ 2000-04-08      81    70    68    67    66    57    54    53
## 4 3 Doors Do~ Loser 2000-10-21      76    76    72    69    67    65    55    59
## 5 504 Boyz    Wobb~ 2000-04-15      57    34    25    17    17    31    36    49
## 6 98^0        Give~ 2000-08-19      51    39    34    26    26    19     2     2
## # i 68 more variables: wk9 <dbl>, wk10 <dbl>, wk11 <dbl>, wk12 <dbl>,
## #   wk13 <dbl>, wk14 <dbl>, wk15 <dbl>, wk16 <dbl>, wk17 <dbl>, wk18 <dbl>,
## #   wk19 <dbl>, wk20 <dbl>, wk21 <dbl>, wk22 <dbl>, wk23 <dbl>, wk24 <dbl>,
## #   wk25 <dbl>, wk26 <dbl>, wk27 <dbl>, wk28 <dbl>, wk29 <dbl>, wk30 <dbl>,
## #   wk31 <dbl>, wk32 <dbl>, wk33 <dbl>, wk34 <dbl>, wk35 <dbl>, wk36 <dbl>,
## #   wk37 <dbl>, wk38 <dbl>, wk39 <dbl>, wk40 <dbl>, wk41 <dbl>, wk42 <dbl>,
## #   wk43 <dbl>, wk44 <dbl>, wk45 <dbl>, wk46 <dbl>, wk47 <dbl>, wk48 <dbl>, ...
\end{verbatim}

\begin{Shaded}
\begin{Highlighting}[]
\NormalTok{?pivot\_longer}
\end{Highlighting}
\end{Shaded}

\begin{Shaded}
\begin{Highlighting}[]
\NormalTok{billboard\_long }\OtherTok{\textless{}{-}}\NormalTok{ billboard }\SpecialCharTok{\%\textgreater{}\%}
\FunctionTok{pivot\_longer}\NormalTok{(}\AttributeTok{cols =} \FunctionTok{starts\_with}\NormalTok{(}\StringTok{"wk"}\NormalTok{),}
             \AttributeTok{names\_to =} \StringTok{"week"}\NormalTok{,}
             \AttributeTok{values\_to =} \StringTok{"position"}\NormalTok{,}
             \AttributeTok{names\_prefix =} \StringTok{"wk"}\NormalTok{, }
             \AttributeTok{values\_drop\_na =} \ConstantTok{TRUE}\NormalTok{)}
\end{Highlighting}
\end{Shaded}

\begin{Shaded}
\begin{Highlighting}[]
\FunctionTok{head}\NormalTok{(billboard\_long)}
\end{Highlighting}
\end{Shaded}

\begin{verbatim}
## # A tibble: 6 x 5
##   artist track                   date.entered week  position
##   <chr>  <chr>                   <date>       <chr>    <dbl>
## 1 2 Pac  Baby Don't Cry (Keep... 2000-02-26   1           87
## 2 2 Pac  Baby Don't Cry (Keep... 2000-02-26   2           82
## 3 2 Pac  Baby Don't Cry (Keep... 2000-02-26   3           72
## 4 2 Pac  Baby Don't Cry (Keep... 2000-02-26   4           77
## 5 2 Pac  Baby Don't Cry (Keep... 2000-02-26   5           87
## 6 2 Pac  Baby Don't Cry (Keep... 2000-02-26   6           94
\end{verbatim}

\begin{Shaded}
\begin{Highlighting}[]
\FunctionTok{dim}\NormalTok{(billboard)}
\end{Highlighting}
\end{Shaded}

\begin{verbatim}
## [1] 317  79
\end{verbatim}

\begin{Shaded}
\begin{Highlighting}[]
\FunctionTok{dim}\NormalTok{(billboard\_long)}
\end{Highlighting}
\end{Shaded}

\begin{verbatim}
## [1] 5307    5
\end{verbatim}

\begin{Shaded}
\begin{Highlighting}[]
\CommentTok{\#look! \_long is actually longer! with more rows}
\end{Highlighting}
\end{Shaded}

\begin{Shaded}
\begin{Highlighting}[]
\NormalTok{billboard\_long}
\end{Highlighting}
\end{Shaded}

\begin{verbatim}
## # A tibble: 5,307 x 5
##    artist  track                   date.entered week  position
##    <chr>   <chr>                   <date>       <chr>    <dbl>
##  1 2 Pac   Baby Don't Cry (Keep... 2000-02-26   1           87
##  2 2 Pac   Baby Don't Cry (Keep... 2000-02-26   2           82
##  3 2 Pac   Baby Don't Cry (Keep... 2000-02-26   3           72
##  4 2 Pac   Baby Don't Cry (Keep... 2000-02-26   4           77
##  5 2 Pac   Baby Don't Cry (Keep... 2000-02-26   5           87
##  6 2 Pac   Baby Don't Cry (Keep... 2000-02-26   6           94
##  7 2 Pac   Baby Don't Cry (Keep... 2000-02-26   7           99
##  8 2Ge+her The Hardest Part Of ... 2000-09-02   1           91
##  9 2Ge+her The Hardest Part Of ... 2000-09-02   2           87
## 10 2Ge+her The Hardest Part Of ... 2000-09-02   3           92
## # i 5,297 more rows
\end{verbatim}

\begin{Shaded}
\begin{Highlighting}[]
\NormalTok{billboard\_long}\SpecialCharTok{\%\textgreater{}\%}
  \FunctionTok{pivot\_wider}\NormalTok{(}\AttributeTok{names\_from =}\NormalTok{ week,}
              \AttributeTok{values\_from =}\NormalTok{ position)}
\end{Highlighting}
\end{Shaded}

\begin{verbatim}
## # A tibble: 317 x 68
##    artist     track date.entered   `1`   `2`   `3`   `4`   `5`   `6`   `7`   `8`
##    <chr>      <chr> <date>       <dbl> <dbl> <dbl> <dbl> <dbl> <dbl> <dbl> <dbl>
##  1 2 Pac      Baby~ 2000-02-26      87    82    72    77    87    94    99    NA
##  2 2Ge+her    The ~ 2000-09-02      91    87    92    NA    NA    NA    NA    NA
##  3 3 Doors D~ Kryp~ 2000-04-08      81    70    68    67    66    57    54    53
##  4 3 Doors D~ Loser 2000-10-21      76    76    72    69    67    65    55    59
##  5 504 Boyz   Wobb~ 2000-04-15      57    34    25    17    17    31    36    49
##  6 98^0       Give~ 2000-08-19      51    39    34    26    26    19     2     2
##  7 A*Teens    Danc~ 2000-07-08      97    97    96    95   100    NA    NA    NA
##  8 Aaliyah    I Do~ 2000-01-29      84    62    51    41    38    35    35    38
##  9 Aaliyah    Try ~ 2000-03-18      59    53    38    28    21    18    16    14
## 10 Adams, Yo~ Open~ 2000-08-26      76    76    74    69    68    67    61    58
## # i 307 more rows
## # i 57 more variables: `9` <dbl>, `10` <dbl>, `11` <dbl>, `12` <dbl>,
## #   `13` <dbl>, `14` <dbl>, `15` <dbl>, `16` <dbl>, `17` <dbl>, `18` <dbl>,
## #   `19` <dbl>, `20` <dbl>, `21` <dbl>, `22` <dbl>, `23` <dbl>, `24` <dbl>,
## #   `25` <dbl>, `26` <dbl>, `27` <dbl>, `28` <dbl>, `29` <dbl>, `30` <dbl>,
## #   `31` <dbl>, `32` <dbl>, `33` <dbl>, `34` <dbl>, `35` <dbl>, `36` <dbl>,
## #   `37` <dbl>, `38` <dbl>, `39` <dbl>, `40` <dbl>, `41` <dbl>, `42` <dbl>, ...
\end{verbatim}

\begin{Shaded}
\begin{Highlighting}[]
\CommentTok{\#error bc we did not add a column name}
\end{Highlighting}
\end{Shaded}

\begin{Shaded}
\begin{Highlighting}[]
\CommentTok{\#separate () separates a column by a specific separator}
\NormalTok{?separate}
\NormalTok{billboard }\SpecialCharTok{\%\textgreater{}\%}
  \FunctionTok{separate}\NormalTok{(}\AttributeTok{col =}\NormalTok{ date.entered, }
           \AttributeTok{into =} \FunctionTok{c}\NormalTok{(}\StringTok{"year"}\NormalTok{, }\StringTok{"month"}\NormalTok{, }\StringTok{"day"}\NormalTok{))}
\end{Highlighting}
\end{Shaded}

\begin{verbatim}
## # A tibble: 317 x 81
##    artist      track year  month day     wk1   wk2   wk3   wk4   wk5   wk6   wk7
##    <chr>       <chr> <chr> <chr> <chr> <dbl> <dbl> <dbl> <dbl> <dbl> <dbl> <dbl>
##  1 2 Pac       Baby~ 2000  02    26       87    82    72    77    87    94    99
##  2 2Ge+her     The ~ 2000  09    02       91    87    92    NA    NA    NA    NA
##  3 3 Doors Do~ Kryp~ 2000  04    08       81    70    68    67    66    57    54
##  4 3 Doors Do~ Loser 2000  10    21       76    76    72    69    67    65    55
##  5 504 Boyz    Wobb~ 2000  04    15       57    34    25    17    17    31    36
##  6 98^0        Give~ 2000  08    19       51    39    34    26    26    19     2
##  7 A*Teens     Danc~ 2000  07    08       97    97    96    95   100    NA    NA
##  8 Aaliyah     I Do~ 2000  01    29       84    62    51    41    38    35    35
##  9 Aaliyah     Try ~ 2000  03    18       59    53    38    28    21    18    16
## 10 Adams, Yol~ Open~ 2000  08    26       76    76    74    69    68    67    61
## # i 307 more rows
## # i 69 more variables: wk8 <dbl>, wk9 <dbl>, wk10 <dbl>, wk11 <dbl>,
## #   wk12 <dbl>, wk13 <dbl>, wk14 <dbl>, wk15 <dbl>, wk16 <dbl>, wk17 <dbl>,
## #   wk18 <dbl>, wk19 <dbl>, wk20 <dbl>, wk21 <dbl>, wk22 <dbl>, wk23 <dbl>,
## #   wk24 <dbl>, wk25 <dbl>, wk26 <dbl>, wk27 <dbl>, wk28 <dbl>, wk29 <dbl>,
## #   wk30 <dbl>, wk31 <dbl>, wk32 <dbl>, wk33 <dbl>, wk34 <dbl>, wk35 <dbl>,
## #   wk36 <dbl>, wk37 <dbl>, wk38 <dbl>, wk39 <dbl>, wk40 <dbl>, wk41 <dbl>, ...
\end{verbatim}

\begin{Shaded}
\begin{Highlighting}[]
\CommentTok{\#separated the dates into year, month, day}
\CommentTok{\#she automatically set the sep(separator) to be an non{-}alphanumeric characters}
\end{Highlighting}
\end{Shaded}

\begin{Shaded}
\begin{Highlighting}[]
\CommentTok{\#billboard\%\textgreater{}\%}
\CommentTok{\#unite(data,col, sep "\_" )}
\end{Highlighting}
\end{Shaded}

\begin{Shaded}
\begin{Highlighting}[]
\CommentTok{\#explicit missing values {-} NA is in the data{-}set}
\CommentTok{\#implicit missing {-} the value just DOES not appear in the data{-}set (you have to know this data)}
\CommentTok{\#complete() {-} turns implicit missing values into explicit missing values }
\CommentTok{\#fill(){-} fills missing values in selected columns using the next or previous entry. }
\CommentTok{\#replace\_NA() {-} Replace NA\textquotesingle{}s with specified values  {-}\textgreater{} 0 instead of NA.}
\CommentTok{\#NA\_if : Convert specified values to NA. }
\end{Highlighting}
\end{Shaded}

\#ICMA Sept 25 2024

\begin{Shaded}
\begin{Highlighting}[]
\NormalTok{head }\OtherTok{=}\NormalTok{(ChickWeight)}
\NormalTok{ChickWeight }\SpecialCharTok{\%\textgreater{}\%}
  \FunctionTok{ggplot}\NormalTok{(}\FunctionTok{aes}\NormalTok{(}\AttributeTok{x =}\NormalTok{ weight)) }\SpecialCharTok{+} \CommentTok{\#provide the aes(aesthetic mapping, in this case, x{-}axis only)}
\FunctionTok{geom\_histogram}\NormalTok{()}
\end{Highlighting}
\end{Shaded}

\begin{verbatim}
## `stat_bin()` using `bins = 30`. Pick better value with `binwidth`.
\end{verbatim}

\includegraphics{r-notebook1-412_files/figure-latex/unnamed-chunk-27-1.pdf}

\begin{Shaded}
\begin{Highlighting}[]
\NormalTok{head }\OtherTok{=}\NormalTok{(ChickWeight)}
\NormalTok{ChickWeight }\SpecialCharTok{\%\textgreater{}\%}
  \FunctionTok{ggplot}\NormalTok{(}\FunctionTok{aes}\NormalTok{(}\AttributeTok{y =}\NormalTok{ weight, }
             \AttributeTok{x =}\NormalTok{ Time,}
             \AttributeTok{color =}\NormalTok{ Diet)) }\SpecialCharTok{+} 
  \FunctionTok{geom\_point}\NormalTok{() }\SpecialCharTok{+}
  \FunctionTok{geom\_smooth}\NormalTok{() }\SpecialCharTok{+}
\CommentTok{\#creates function of best fit}
\CommentTok{\# to remove the error bands, write the following = geom\_smooth(se = FALSE)}
\FunctionTok{theme\_minimal}\NormalTok{()}
\end{Highlighting}
\end{Shaded}

\begin{verbatim}
## `geom_smooth()` using method = 'loess' and formula = 'y ~ x'
\end{verbatim}

\includegraphics{r-notebook1-412_files/figure-latex/unnamed-chunk-28-1.pdf}
\#ICMA Oct 7 2024

\begin{Shaded}
\begin{Highlighting}[]
\FunctionTok{as.numeric}\NormalTok{(}\StringTok{"12"}\NormalTok{)}
\end{Highlighting}
\end{Shaded}

\begin{verbatim}
## [1] 12
\end{verbatim}

\begin{Shaded}
\begin{Highlighting}[]
\CommentTok{\#only parses digits, words do not count, but numbers as chars work tho.}
\end{Highlighting}
\end{Shaded}

We can use is.numeric -\textgreater{} aka is.TYPE() in order to do a
check to see i she is wo she is. Doubles are r's default type of
numeric. ``as.integer()'' forces a type to become an integer. To check
type, we can do typeof(), to get the type directly.

\begin{Shaded}
\begin{Highlighting}[]
\FunctionTok{as.integer}\NormalTok{(}\FloatTok{12.3}\NormalTok{)}
\end{Highlighting}
\end{Shaded}

\begin{verbatim}
## [1] 12
\end{verbatim}

True = 1 False = 0

TRY on your own activity

\begin{Shaded}
\begin{Highlighting}[]
\DecValTok{3} \SpecialCharTok{\textless{}} \DecValTok{5} \SpecialCharTok{|} \ConstantTok{FALSE}
\end{Highlighting}
\end{Shaded}

\begin{verbatim}
## [1] TRUE
\end{verbatim}

\begin{Shaded}
\begin{Highlighting}[]
\CommentTok{\#true}
\end{Highlighting}
\end{Shaded}

\begin{Shaded}
\begin{Highlighting}[]
\FunctionTok{is.numeric}\NormalTok{(}\FunctionTok{mean}\NormalTok{(}\FunctionTok{c}\NormalTok{(}\DecValTok{12}\NormalTok{,}\DecValTok{31}\NormalTok{,}\FloatTok{15.57}\NormalTok{,}\DecValTok{4}\NormalTok{)))}
\end{Highlighting}
\end{Shaded}

\begin{verbatim}
## [1] TRUE
\end{verbatim}

\begin{Shaded}
\begin{Highlighting}[]
\CommentTok{\#true 2nd}
\end{Highlighting}
\end{Shaded}

\begin{Shaded}
\begin{Highlighting}[]
\NormalTok{((}\DecValTok{1}\SpecialCharTok{+}\DecValTok{2}\NormalTok{)}\SpecialCharTok{\textless{}} \SpecialCharTok{{-}}\DecValTok{1000}\NormalTok{) }\SpecialCharTok{|}\NormalTok{ (}\ConstantTok{TRUE} \SpecialCharTok{|} \ConstantTok{NA}\NormalTok{)}
\end{Highlighting}
\end{Shaded}

\begin{verbatim}
## [1] TRUE
\end{verbatim}

\#ICMA Oct 9 2024 Data Structures : lists to get elements *you can do a
list of vectors, and then go 2 brackets deeper to choose an element of
that list

\begin{Shaded}
\begin{Highlighting}[]
\NormalTok{my\_list }\OtherTok{\textless{}{-}} \FunctionTok{list}\NormalTok{(}\FunctionTok{c}\NormalTok{(}\StringTok{"p"}\NormalTok{,}\StringTok{"o"}\NormalTok{,}\StringTok{"w"}\NormalTok{,}\StringTok{"e"}\NormalTok{,}\StringTok{"r"}\NormalTok{))}
\NormalTok{the\_vector\_but\_w }\OtherTok{\textless{}{-}}\NormalTok{ my\_list[[}\DecValTok{1}\NormalTok{]][}\DecValTok{3}\NormalTok{]}
\NormalTok{the\_vector\_but\_w}
\end{Highlighting}
\end{Shaded}

\begin{verbatim}
## [1] "w"
\end{verbatim}

Tibbles

treat like a data frame (table thingie) *tribble for ``row-wise''
declaration

\begin{Shaded}
\begin{Highlighting}[]
\NormalTok{new\_tibble\_columnwise }\OtherTok{\textless{}{-}} \FunctionTok{tibble}\NormalTok{( }\AttributeTok{x  =} \DecValTok{1}\SpecialCharTok{:}\DecValTok{5}\NormalTok{ , }\AttributeTok{y=}\DecValTok{1}\NormalTok{, }\AttributeTok{z =}\NormalTok{ x}\SpecialCharTok{\^{}}\DecValTok{2} \SpecialCharTok{+}\NormalTok{ y)}
\NormalTok{new\_tibble\_columnwise}
\end{Highlighting}
\end{Shaded}

\begin{verbatim}
## # A tibble: 5 x 3
##       x     y     z
##   <int> <dbl> <dbl>
## 1     1     1     2
## 2     2     1     5
## 3     3     1    10
## 4     4     1    17
## 5     5     1    26
\end{verbatim}

Combining Data : diff vars - use join left\_join vs right\_join

left join -\textgreater{} joiner gets added to og, but only to stuff
that they share in common with og (columnwise) right join
-\textgreater{} joiner becomes the joinee, and joiner values are matched
to og.

full join -\textgreater{} combines everybody

inner join -\textgreater{} venn diagram of rows that match

\begin{Shaded}
\begin{Highlighting}[]
\NormalTok{age\_data }\OtherTok{\textless{}{-}} \FunctionTok{tribble}\NormalTok{(}
                     \SpecialCharTok{\textasciitilde{}}\NormalTok{id, }\SpecialCharTok{\textasciitilde{}}\NormalTok{age,}
                     \DecValTok{1}\NormalTok{,}\DecValTok{8}\NormalTok{,}
                     \DecValTok{2}\NormalTok{,}\DecValTok{10}\NormalTok{,}
                     \DecValTok{3}\NormalTok{,}\DecValTok{8}\NormalTok{,}
                     \DecValTok{5}\NormalTok{,}\DecValTok{9}
                     
\NormalTok{              )}

\NormalTok{gender\_data }\OtherTok{\textless{}{-}} \FunctionTok{tribble}\NormalTok{ (}
                        \SpecialCharTok{\textasciitilde{}}\NormalTok{id, }\SpecialCharTok{\textasciitilde{}}\NormalTok{ gender,}
                        \DecValTok{1}\NormalTok{,}\StringTok{"f"}\NormalTok{,}
                        \DecValTok{2}\NormalTok{,}\StringTok{"m"}\NormalTok{,}
                        \DecValTok{3}\NormalTok{,}\StringTok{"nb"}\NormalTok{,}
                        \DecValTok{4}\NormalTok{,}\StringTok{"m"}\NormalTok{,}
                        \DecValTok{6}\NormalTok{,}\StringTok{"f"}\NormalTok{,}
\NormalTok{              )}
\end{Highlighting}
\end{Shaded}

\begin{Shaded}
\begin{Highlighting}[]
\NormalTok{age\_data}
\end{Highlighting}
\end{Shaded}

\begin{verbatim}
## # A tibble: 4 x 2
##      id   age
##   <dbl> <dbl>
## 1     1     8
## 2     2    10
## 3     3     8
## 4     5     9
\end{verbatim}

\begin{Shaded}
\begin{Highlighting}[]
\NormalTok{gender\_data}
\end{Highlighting}
\end{Shaded}

\begin{verbatim}
## # A tibble: 5 x 2
##      id gender
##   <dbl> <chr> 
## 1     1 f     
## 2     2 m     
## 3     3 nb    
## 4     4 m     
## 5     6 f
\end{verbatim}

\begin{Shaded}
\begin{Highlighting}[]
\FunctionTok{full\_join}\NormalTok{(age\_data, gender\_data)}
\end{Highlighting}
\end{Shaded}

\begin{verbatim}
## Joining with `by = join_by(id)`
\end{verbatim}

\begin{verbatim}
## # A tibble: 6 x 3
##      id   age gender
##   <dbl> <dbl> <chr> 
## 1     1     8 f     
## 2     2    10 m     
## 3     3     8 nb    
## 4     5     9 <NA>  
## 5     4    NA m     
## 6     6    NA f
\end{verbatim}

intersect is different from inner join intersect = a set operation
-\textgreater{} this means that interesct worls if the data in in two
different sets are the same there is also : union, (rows that appear in
ether or both y and z) setdiff(rows that appear in y but not in z)

bind\_rows() bindcols() -\textgreater{} both assumes that everything is
in the same order.(APPEND)

\#ICMA Oct 16 2024

\begin{Shaded}
\begin{Highlighting}[]
\CommentTok{\# for loops ( and iteration in R)}
\CommentTok{\#you should write }
\CommentTok{\#for (var in seq)\{}
  \CommentTok{\#expr}
\CommentTok{\#\}}
\CommentTok{\#for if statements, the if/elseif are the same syntax as java}
\end{Highlighting}
\end{Shaded}

\#ICMA Oct 30 2024

\begin{Shaded}
\begin{Highlighting}[]
\CommentTok{\#Where would you find the YAML header ( top of the file (YAML aint markup language)) "header info" }
\CommentTok{\#plain text( no frills, no bolding (which is what rich text is))}

\CommentTok{\#what is knitting? }
\CommentTok{\#sends rmd {-}\textgreater{} md }

\CommentTok{\#error from yaml toc : Error in yaml::yaml.load(..., eval.expr = TRUE) : }
\CommentTok{\#Scanner error: mapping values are not allowed in this context at line 6, column 8}
\CommentTok{\#Calls: \textless{}Anonymous\textgreater{} ... parse\_yaml\_front\_matter {-}\textgreater{} yaml\_load {-}\textgreater{} \textless{}Anonymous\textgreater{}}
\CommentTok{\#Execution halted}

\CommentTok{\#after setting echo to false, (within knitr(download pkg), should see no code chunks, only outputs)}

\CommentTok{\#Markdown text: }
\CommentTok{\#use *italics*, **bold**, }

\CommentTok{\#fig.cap, allows us to add captions to our figures. }
\end{Highlighting}
\end{Shaded}

Let's write a sentence that has \textbf{bold} and \emph{italics} Part 2
: create a var, then write a sentence that uses that variable, bold, and
italics.

\begin{Shaded}
\begin{Highlighting}[]
\NormalTok{name }\OtherTok{\textless{}{-}}  \StringTok{"Madison"}
\end{Highlighting}
\end{Shaded}

My name is Madison!

\begin{Shaded}
\begin{Highlighting}[]
\NormalTok{ChickWeight }\SpecialCharTok{\%\textgreater{}\%}
  \FunctionTok{ggplot}\NormalTok{(}\FunctionTok{aes}\NormalTok{(}\AttributeTok{y =}\NormalTok{ weight, }\AttributeTok{x =}\NormalTok{ Time, }\AttributeTok{group =}\NormalTok{ Chick)) }\SpecialCharTok{+} 
  \FunctionTok{geom\_line}\NormalTok{()}
\end{Highlighting}
\end{Shaded}

\begin{figure}
\centering
\includegraphics{r-notebook1-412_files/figure-latex/chick-plot-1.pdf}
\caption{Chick weight across time}
\end{figure}

\end{document}
